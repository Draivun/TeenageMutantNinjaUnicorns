\hypertarget{index_unicorn}{}\section{Unicorn}\label{index_unicorn}
The unicorn in the game can be created by creating an object of the \hyperlink{classunicorn}{unicorn} class. The image of the unicorn is created by the \hyperlink{classimage__from__file}{image\+\_\+from\+\_\+file} class. It also uses the \hyperlink{classphysics}{physics} class to create a more realistic jumping and falling motion. The class dus also inherrit \hyperlink{classdrawable}{drawable}. Drawable comes with his predefined funtions that can be used by the unicorn due to the inherritance. The unicorn class also overrides the couple of abstract functions. These functions are \hyperlink{classdrawable_a4e49e2c1121704c83ce24c5f48dd910f}{drawable\+::draw()}, \hyperlink{classdrawable_ad0d3930c045cc6776aa2c3965be32491}{drawable\+::move()}, \hyperlink{classdrawable_ac39691470b7874f5dec59efe649d3981}{drawable\+::jump()}, \hyperlink{classdrawable_ae013ac0be47538be9ce885d6642daf73}{drawable\+::get\+Global\+Bounds()}, \hyperlink{classdrawable_a715df01a318331e5611a2b0ad30109ff}{drawable\+::run\+\_\+actions()}, \hyperlink{classdrawable_abbc6e0089d502ba48c3fcb9c96e3966e}{drawable\+::check\+\_\+for\+\_\+collisions}. The \hyperlink{classunicorn_a570c34d5669a8d2a61bdc1481e6f9dee}{unicorn\+::draw()} function first checks if the direction the unicorn is facing is correct. If it is not, the unicorn is turned around. After that either the \hyperlink{classphysics_aaf1c57aa6e35b9c83ccbfdfa8c18468c}{physics\+::jumping()} function is called to move the unicorn up or the \hyperlink{classphysics_acca1ee2fb8b760b6e4ee61ae7c2ee3da}{physics\+::falling()} function is called to move the unicorn down. There is one exception to both, the unicorn dus not fall if he is on the ground and the unicorn also dus not go up when something is above him. The \hyperlink{classunicorn_a162f200a68342f7bc0baaf17c8cf3f9f}{unicorn\+::move()} function moves the unicorn right and left unless there is a collision on the side he is trying to move to. The \hyperlink{classunicorn_a07d5ca4e66632c0e871221a27146805a}{unicorn\+::jump()} function sets the counter for jumping to 25 so it can be counted down to 3 in the draw function. These values seemed to give us the most realistic jumping effect This counter is used to calculate the speed with wich the unicorn goes up in the \hyperlink{classphysics_aaf1c57aa6e35b9c83ccbfdfa8c18468c}{physics\+::jumping()} function. It dus not do this when he is not on the ground or when the counter is not 0. This way dubble jumping is impossible. \hyperlink{classunicorn_a1bac09fc59b04f14f5a093bc4daa04da}{unicorn\+::get\+Global\+Bounds()} returns the global bounds of the image. These global bounds can be used for collision detection. The \hyperlink{classunicorn_aadb47a9981c46d6add8704074df117df}{unicorn\+::run\+\_\+actions()} function goes through the list of actions. It calls the \hyperlink{classaction_ab4f8d0f7552450455977d09a889c18c7}{action\+::operator()()} from the \hyperlink{classaction}{action} class on all the actions and the rest is handled by the operator() function.\hypertarget{index_wall}{}\section{The walls of the game}\label{index_wall}
The wall creates a rectangle on the screen on specified position and with a specified size. The color of the wall can also be set to an initial color. The class inherrits \hyperlink{classdrawable}{drawable} to be able to use all its functionality. The functions from drawable that are redefined here are\+: \hyperlink{classdrawable_a4e49e2c1121704c83ce24c5f48dd910f}{drawable\+::draw()}, \hyperlink{classdrawable_ae013ac0be47538be9ce885d6642daf73}{drawable\+::get\+Global\+Bounds()}, \hyperlink{classdrawable_a2ed0f8bb53f33477f7722efa7bb24583}{drawable\+::object\+\_\+information()}. The \hyperlink{classwall_aa25b8377e1d9a209fabd2271294f05d0}{wall\+::draw()} function draws the object and sets the color and position. The \hyperlink{classwall_a317a464c879cfdf9464bd6f1b62d9101}{wall\+::get\+Global\+Bounds()} function returns the global bounds of the rectangle. The \hyperlink{classwall_aab1de4f144f176b134a967ba08747932}{wall\+::object\+\_\+information()} function returns the information of the object as \href{http://www.cplusplus.com/reference/string/string/string/}{\tt std\+::string}. This includes the size and the color. The rest of the information comes from the \hyperlink{classdrawable_a2ed0f8bb53f33477f7722efa7bb24583}{drawable\+::object\+\_\+information()} function.\hypertarget{index_background}{}\section{the looks of it}\label{index_background}
The background of the game. the background creates a sprite that is covered troughout the gamefield creating a background for the game. the background is een inheratance of the \hyperlink{classdrawable}{drawable} class and uses the \hyperlink{classdrawable_a4e49e2c1121704c83ce24c5f48dd910f}{drawable\+::draw()} function for drawing the background. the background class contains a image object where the background is in.\hypertarget{index_images}{}\section{images}\label{index_images}
All the images just in the game are shown as object of the image class. the image class has multiple function that can be used. the image class is a inherintance of the drawable class. image uses also a override of the \hyperlink{classdrawable_a4e49e2c1121704c83ce24c5f48dd910f}{drawable\+::draw()}. Furthermore the class contains an \#image\+::set\+\_\+size() and an \#image\+::set\+\_\+position() these functions are for setting a new size or position.\+Finally it contains a function for getting the global bounds called \#image\+::get\+Global\+Bounds() and an \#image\+::set\+Texture\+Rect() for setting a texture in the image also this function is beeing used for mirroring a image without position change.\hypertarget{index_drawable}{}\section{drawable}\label{index_drawable}
The superclass drawable is the main class for all objects that will be shown on the window. it contains therefor only virtual functions. the \hyperlink{classdrawable_a4e49e2c1121704c83ce24c5f48dd910f}{drawable\+::draw()} function is de virtual function for drawing a object on a window, further it contains a \hyperlink{classdrawable_ad0d3930c045cc6776aa2c3965be32491}{drawable\+::move(sf\+::\+Vector2f delta)} function that in default moves the position with delta. For finding out if a drawable object is in collsion with a other drawable object there is a collapse function this function calculates if one of the outherlines are crossing with a outherline of the other drawable if so there is collision found and the side of collsion will be put in a struct called \hyperlink{structcollision}{collision}. For calculating if a outherline is crossing the other outherline the class contains the \hyperlink{classdrawable_a0d3278e4e888fc8289468e8893dd8329}{drawable\+::within()} and the \hyperlink{classdrawable_ab5c0e1af885f214bc9ef0da47cdb5ac9}{drawable\+::within\+\_\+range()} function these fucntion looks if a certain point is in a given range. furthermore the class contains two functions \+: \hyperlink{classdrawable_a2ed0f8bb53f33477f7722efa7bb24583}{drawable\+::object\+\_\+information()} and \hyperlink{classdrawable_add3d8569fe2616ae0ed503b19c92c08e}{drawable\+::string\+\_\+from\+\_\+color()}, these functions are for returning the information of the object in a \href{http://www.cplusplus.com/reference/string/string/string/}{\tt std\+::string}\hypertarget{index_actions}{}\section{actions}\label{index_actions}
The \hyperlink{classaction}{action} class is a class made for handeling in game actions it contains couple of different constructors that are made for different actions. for using keybaord and mousse there are two different constructors and there is a constructor that can be used in a template way where if parameter 1 gives true the function given in parameter number 2 will be runt.\hypertarget{index_physics}{}\section{physics}\label{index_physics}
The action class is a class that contains the needed functions for making movement in the game feel natural. for jumping there is the \hyperlink{classphysics_aaf1c57aa6e35b9c83ccbfdfa8c18468c}{physics\+::jumping()} function and for the grafity there is the \hyperlink{classphysics_acca1ee2fb8b760b6e4ee61ae7c2ee3da}{physics\+::falling()} function.\hypertarget{index_camera}{}\section{camera}\label{index_camera}
the camera class has one thing to do and has also only one function and that is following a object of the \hyperlink{classunicorn}{unicorn} class 